% !TeX spellcheck = es_MX
\documentclass[12pt,reqno,letter]{article}
\usepackage[spanish, mexico]{babel}
\usepackage[utf8]{inputenc}
\usepackage[T1]{fontenc}
\usepackage{graphicx} %allows you to use jpg or png images. PDF is still recommended
\usepackage[colorlinks=False]{hyperref} % add links inside PDF files
\usepackage{amsmath}  % Math fonts
\usepackage{amsfonts} %
\usepackage{amssymb}  %
\usepackage{multicol}
\usepackage{hyperref} % agrega links
%% Sets page size and margins. You can edit this to your liking
\usepackage[top=1.3cm, bottom=2.0cm, outer=2.5cm, inner=2.5cm, heightrounded,
marginparwidth=1.5cm, marginparsep=0.4cm, margin=2.5cm]{geometry}
%% \usepackage[authoryear]{natbib}
\usepackage[affil-it]{authblk}
\bibliographystyle{abbrvnat}
\usepackage{enumitem}
\providecommand{\tightlist}{%
	\setlength{\itemsep}{0pt}\setlength{\parskip}{0pt}}
%\PrerenderUnicode{ü}

\begin{document}
	\title{ Implementación del Método de Longstaff \& Schwartz para valuación de opciones americanas }
	\author{Jorge III Altamirano Astorga (175904), Eduardo Selim Martínez Mayorga (175921), Ariel Ernesto Vallarino Maritorena (175875)}
	\affil{Instituto Tecnológico Autónomo de México}
	\date{22 de Mayo, 2019}
	\maketitle
	
	\begin{abstract}
		Proyecto donde utilizamos la técnica aplicada en la investigación de Longstaff-Schwartz (2001, UCLA) denominada  “Aproximación por mínimos cuadrados Monte Carlo” (LSM, por sus siglas en inglés) para evaluar opciones americanas con múltiples factores con el fin de estimar el “payoff” donde tradicionalmente no se podían utilizar diferencias finitas con el fin obtener rendimientos
	\end{abstract}

	\tableofcontents

	
	%%\begin{multicols}{2}

	\section{Introducción}
	Utilizaremos el técnica detallada en la investigación de Longstaff-Schwartz (2001, UCLA) denominada  “Aproximación por mínimos cuadrados Monte Carlo” (LSM, por sus siglas en inglés) para evaluar opciones americanas con múltiples factores con el fin de estimar el “payoff” donde tradicionalmente no se podían utilizar diferencias finitas.
	
	\subsection{Características}
	\begin{itemize}
		\item Es difícil estimar el “payoff” en situaciones donde la opción es afectada por más de 1 factor; esto es debido a que el método de diferencia finita y binomial son imprácticos con múltiples factores. Es de notarse que Wall Street utiliza diferencia finita sobre simplificando a 1 factor, aún cuando es demostrable que la opción es afectada por más de un factor.
		\item La simulación es una alternativa prometedora a diferencias finitas. 
		\item Esta técnica es útil para distintas opciones, entre las cuales se incluyen: hipotecas, forex, commodities, seguros, energía, swap, mercados emergentes, deuda soberana, convertibles.
		\item La simulación permite que las variables de estado sigan un proceso estocástico general, como “jump diffusions”.
		\item Funciona con semi martingalas
		\item Las simulaciones tienen las siguientes características: simples, paralelizables, transparentes y flexibles.
			\subitem Paralelizable: trabajan bien en ambientes computacionales paralelos: esto ayuda al tipo de escalabilidad que buscamos tener en el presente trabajo.
			\subitem Simpleza: el único método a implementar es el simple método de Mínimos Cuadrados.
		
		\item Este enfoque es intuitivo al ser determinado por la función de la esperanza condicional del payoff inmediato versus continuar ejerciendo la opción. Dicha condicional se puede estimar con el cruce seccional de información en la simulación utilizando mínimos cuadrados.
		Obteniendo la función de la esperanza condicional para cada fecha del ejercicio se podrá obtener la especificación óptima para ejercer cada trayectoria (“path”).
		
		En contraste investigaciones previas, desde Tilley (1993) hasta García (1999); no se utiliza Longstaff-Schwartz proponen no utilizar varias estratificaciones o técnicas de parametrización para aproximar la función de densidad de transición o acotar el ejercicio.
		\item Longstaff-Schwartz utiliza la estrategia de enfocarse directamente en la función de Esperanza Condicional. Otros artículos de investigación también han adoptado este enfoque. Sin embargo, Longstaff-Schwartz nos gustó por ser un enfoque más pragmático, y más eficiente computacionalmente. Esto es debido a que solo realiza regresiones sobre los paths que pueden ser monetizados en la opción. 
		\item Longstaff-Schwartz proponen un enfoque con buenos resultados de rendimiento y desempeño comparándolo con otras técnicas
		\item Nuestra principal referencia menciona cuatro artículos con distintos niveles de complejidad que muestran ilustrativamente diferentes escenarios de opciones que demuestran el método y su efectividad.
	\end{itemize}

	\subsection{Definiciones preliminares}\label{definiciones-preliminares}
	
	\textbf{Definición:} Un contrato forward es un contrato entre dos partes
	en el que una de ellas se compromete a comprar y la otra a vender un
	activo dado, en una fecha futura establecida y a un precio establecido.
	
	\begin{itemize}
		\tightlist
		\item
		Definición de call
	\end{itemize}
	
	\textbf{Definición:} Una opción call (larga) es un contrato que le da al
	poseedor de la opción, el derecho más no la obligación de comprar un
	activo dado, en una fecha futura establecida y a un precio establecido.
	
	Si \(K\) es el precio preestablecio (que se conoce como precio
	\emph{strike}), \(S_t\) es el precio del activo al tiempo \(t\) y \(T\)
	es la fecha futura establecida en el contrato (que se conoce como fecha
	de vencimiento del contrato), entonces el poseedor de la call ejercera
	su derecho a comprar sólo si \(K < S_T\), i.e.~si en el mercado es
	costoso el activo con respecto al precio \emph{strike}.
	
	A diferencia del contrato forward, bajo un contrato call una de las
	partes nunca pierde (el comprador del contrato) pero la otra tiene una
	pérdida tentativa, es por esto que este tipo de contratos tiene un
	precio (que se le conoce como prima de la call). Uno de los objetivos de
	la teoría de valuacion de opciones es determinar un precio razonable
	para éstas.
	
	Al vendededor de un contrato call se le conoce como parte corta de la
	call y éste tiene la obligación de vender el activo en caso de que la
	parte larga de la call quiere ejercer su derecho a comprarla.
	
	\textbf{Definición:} Una opción put larga es un contrato que le da al
	poseedor de la opción, el derecho más no la obligación de vender un
	activo dado, en una fecha futura establecida y a un precio establecido.
	
	Si \(K\) es el precio preestablecio (que se conoce como precio
	\emph{strike}), \(S_t\) es el precio del activo al tiempo \(t\) y \(T\)
	es la fecha futura establecida en el contrato (que se conoce como fecha
	de vencimiento del contrato), entonces el poseedor de la put ejercera su
	derecho a vender sólo si \(K > S_T\), i.e.~si en el mercado es barato el
	activo con respecto al precio \emph{strike}.
	
	A diferencia del contrato forward, bajo un contrato put una de las
	partes nunca pierde (el comprador del contrato) pero la otra tiene una
	pérdida tentativa, es por esto que este tipo de contratos tiene un
	precio (que se le conoce como prima de la put). Uno de los objetivos de
	la teoría de valuacion de opciones es determinar un precio razonable
	para éstas.
	
	Al vendededor de un contrato put se le conoce como parte corta de la put
	y éste tiene la obligación de comprar el activo en caso de que la parte
	larga de la put quiere ejercer su derecho a venderla.
	
	\subsubsection{Estilo de opciones: Europeo, Americano ó
		Bermuda}\label{estilo-de-opciones-europeo-americano-o-bermuda}
	
	\begin{itemize}
		\item
		Se dice que una opción tiene estilo Europeo si sólo se puede ejercer
		en la fecha de vencimiento.
		\item
		Se dice que una opción tiene estilo Americano si se puede ejercer en
		cualquier momento previo a la fecha de vencimiento (incluyéndola).
		\item
		Se dice que una opción tiene estilo Bermunda si sólo se puede ejercer
		determinadas fechas previas a la fecha de vencimiento (incluyéndola).
	\end{itemize}

	\section{Desarrollo}
	
	\subsection{Estimación de Parámetros}
	Sea $(S_t)_{\geq 0}$ un proceso estocástico que representa el precio de un activo, i.e. $S_t$ es el precio de la acción al tiempo $t$.
	
	Se dice que el proceso $(S_t)_{\geq 0}$ es un movimiento Browniano geométrico con parámetros $\mu,\sigma>0$ si es la solución de la ecuación diferencial estocástica
	$$dS_t = \mu S_t dt + \sigma S_t dW_t$$
	
	para algún valor inicial $S_0$.
	Esta expresión se puede reescribir como
	$$\frac{dS_t}{S_t} = \mu dt + \sigma dW_t$$
	
	Se puede dar una interpretación de la ecuación anterior de la siguiente manera
	$$\frac{S_{t+dt}-S_t}{S_t}=\frac{dS_t}{S_t} = \mu dt + \sigma dW_t = \mbox{contribución determinista + constribución estocástica}$$
	
	donde se supone que la contribución determinista es proporcional y la parte estocástica tiene ley Gaussiana.
	
	A la constante $\mu$ se le conoce como drift del proceso y $\sigma$ se conoce  como parámetro volatilidad o de difusión.
	
	Se puede demostrar que una solución explícita para la ecuación diferencial estocástica anterior es
	$$S_t =  S_0 \exp\left\{\alpha t + \sigma W_t\right\} =  S_0 \exp\left\{\mu t -\frac{1}{2}\sigma^2t + \sigma W_t\right\},$$
	donde $\alpha = \mu-\frac{1}{2}\sigma^2$.
	
	Observación:
	Si $\sigma = 0$, i.e. no hay ruido estocástico, entonces la ecuación se convierte en
	$$\frac{dS_t}{S_t} = \mu dt,$$
	equivalentemente
	$$\frac{d}{dt}\log(S_t) = \mu$$
	y por lo tanto
	$$S_t = S_0 e^{\mu t}$$
	
	Nótese que la diferencia entre la solución determinista y no determinista es el término $\sigma W_t-\frac{1}{2}\sigma^2t$.
	
	A partir de la solución $$S_t = S_0 \exp\left\{\mu t -\frac{1}{2}\sigma^2t + \sigma W_t\right\}$$
	se puede ver que $S_t$ es la exponencial de un movimiento Browniano, i.e. es la exponencial de una variable aleatoria con distribución normal. Equivalentemente $S_t$ tiene distribución log-normal. Esta es una de las razones por las que el movimiento Browniano es adecuado para aplicaciones financieras.
	
	Considérese $t_0=0 < t_1 < t_2 <\ldots<t_n=T$ puntos en el horizonte de tiempo $[0,T]$. Defínase $Y_1,\ldots,Y_n$ como
	
	$$Y_i = \frac{S_{t_i}-S_{t_{i-1}}}{S_{t_{i-1}}}=\frac{S_{t_i}}{S_{t_{i-1}}}-1$$
	
	Si se considera la expansión de Taylor de la función $\log(1+z)$, se tiene que que para $z$ suficientemente pequeña
	$$\log(1+z) = z-\frac{1}{2}z^2 + o(z) \approx z$$
	Entonces si $X_i = 1+ Y_i$, entonces
	$$\log(X_i) = \log(1+Y_i) \approx Y_i = \log(S_{t_i})-\log(S_{t_{i-1}})$$
	Esto significa que los rendimientos exactos son casi idénticos a los log-rendimientos
	$$X_i = \log(S_{t_i})-\log(S_{t_{i-1}}),\ i\in\{1,\ldots,n\}$$
	
	Sea $\Delta t = t_i - t_{i-1}$ para $i\in\{1,\ldots,n\}$. Generalmente $\Delta t = \frac{1}{252}$ si se consideran periodos anuales.
	
	$$X_i + \ldots + X_{i+n-1}=\sum_{k=i}^{i+n-1}[\log(S_{t_i})-\log(S_{t_{i-1}})]=\log(S_{i+n-1})-\log(S_{i-1})$$
	
	Finalmente
	$$X_i = \log(S_{t_i})-\log(S_{t_{i-1}}) = \log\left(\frac{S_{t_i}}{S_{t_{i-1}}}\right)$$
	$$=\alpha \Delta t + \sigma[W_{t_i}-W_{t_{i-1}}]\sim N(\alpha t , \sigma^2 t)$$
	Es decir,
	$$X_i = \log\left(\frac{S_{t_i}}{S_{t_{i-1}}}\right) = \alpha \Delta t + \sigma\sqrt{\Delta t}Z,$$
	donde $Z\sim N(0,1)$
	
	además, para $i\neq j$ $X_{t_i}$ es independiente de $X_{t_j}$ pues cada una depende de incrementos disjuntos del movimiento Browniano estándar.
	
	Estimación de parámetros
	
	Gracias al análisis anterior se puede considerar a $X_1,\ldots,X_n$ como variables aleatorias independientes, idénticamente distribuidas $N(\alpha\Delta t,\sigma^2t \Delta t)$. Los estimadores máximo verosímiles para $\alpha$ y $\sigma^2$ son
	$$\widehat{\alpha}=\frac{1}{\Delta t}\frac{1}{n}\sum_{i=1}^n X_i$$
	$$\widehat{\sigma}^2 = \frac{1}{\Delta t}\frac{1}{n-1}\sum_{i=1}^n(X_i-\bar{X})^2=\frac{\widehat{S}^2}{\Delta t}$$
	
	De aquí que un estimador para $\mu$ es
	$$\widehat{\mu} = \widehat{\alpha} + \frac{1}{2}\widehat{\sigma}^2$$
	
	Observación:
	Nótese que 
	$$\sum_{i=1}^n X_i = \sum_{i=1}^n [\log(S_{t_i})-\log(S_{t_{i-1}})]=\log(S_{t_n})-\log(S_0)$$
	
	De aquí que
	
	$$\widehat{\alpha}=\frac{1}{n\Delta t}[\log(S_{t_n})-\log(S_0)]$$
	
	y entonces
	
	$$\widehat{\mu} =\frac{\log(S_{t_n})-\log(S_0)}{n\Delta t}+ \frac{1}{\Delta t}\frac{1}{n-1}\sum_{i=1}^n(X_i-\bar{X})^2=\frac{\widehat{S}^2}{\Delta t}$$
	$$\widehat{\sigma}^2 = \frac{1}{\Delta t}\frac{1}{n-1}\sum_{i=1}^n(X_i-\bar{X})^2=\frac{\widehat{S}^2}{\Delta t}$$
	
	\section{Valuaciones de opciones
		Europeas}\label{valuaciones-de-opciones-europeas}
	
	Principio de valuación
	
	\subsection{Paridad put-call}\label{paridad-put-call}
	
	Sean \(C(S,K,T)\) y \(P(S,K,T)\) los precios de una call y put Europeas,
	sobre el mismo activo, con el mismo strike y la misma fecha de
	vencimiento. Entonces se satisface una relación entre dichos precios que
	se conoce como la paridad put-call, que se puede plantear en términos
	matemáticos como \[e^{-rT}F_{0,T} + P(S,K,T) = Ke^{rT} + C(S,K,T),\]
	donde
	
	\begin{itemize}
		\tightlist
		\item
		\(F_{0,T} = e^{rT}S_0\) si el activo no paga dividendos.
		\item
		\(F_{0,T} = e^{rT}\left(S_0 - \sum_{j=1}^m Div_j e^{-r t_j}\right)\)
		si el activo paga dividendos discretos.
		\item
		\(F_{0,T} = e^{(r-\delta)T}S_0\) si el activo no paga dividendos
		continuos.
	\end{itemize}
	
	Por ejemplo, para una acción que no paga dividendos, la paridad put-call
	se escribe como
	
	\[S_0 + P(S,K,T) = Ke^{rT} + C(S,K,T),\]
	
	que se puede interpretar fácilmente: si se adquiere una put, se está
	comprando el derecho a vender, por lo tanto se debe contar con el activo
	(\(S_0\)) para poder venderlo en caso de que así se quiera. Por otro
	lado, si se adquiere una call, se está comprando el derecho a comprar,
	por lo tanto eventualmente se podría requerir efectivo \(K\) (por tanto
	se invierten sin riesgo \(Ke^{rT}\)).
	
	\subsection{Fórmula de Black \& Scholes para opciones
		Europeras}\label{formula-de-black-scholes-para-opciones-europeras}
	
	Para acciones que pagan dividendos continuos, las fórmulas de valuación
	para opciones Europeas de Black \& Scholes establecen que el precio de
	una call Europea es
	\[C(S,K,T,\sigma,r,\delta) = S_0e^{-\delta T}\Phi(d_1) - Ke^{-rT}\Phi(d_2)\]
	y el precio de una put Europea es
	\[P(S,K,T,\sigma,r,\delta) = Ke^{-rT}\Phi(-d_2) - S_0e^{-\delta T}\Phi(d_1),\]
	
	con
	\[d_1 = \frac{\log(S_0/K) + (r-\delta+\frac{1}{2}\sigma^2)T}{\sigma\sqrt{T}},\]
	y \[d_2 = d_1 - \sigma\sqrt{T},\] donde \(r\) es la tasa libre de riesgo
	con composición continua, \(\sigma\) es la volatilidad del precio del
	activo subyacente, y \(\delta\) es la tasa de dividendos continuos.
	
	\section{Valuación de opciones
		Americanas}\label{valuacion-de-opciones-americanas}
	
	\subsection{Valuación de opciones Americanas vía la implementación de
		Longstaff \&
		Schwarz}\label{valuacion-de-opciones-americanas-via-la-implementacion-de-longstaff-schwarz}
	
	Se supondrá un espacio de probabilidad filtrado completo
	\((\Omega, \mathcal{F}, \{\mathcal{F}_t\},\mathbb{P})\) y un horizonte
	de tiempo \([0,T]\), donde \(\Omega\) es el conjunto de todas las
	posibles realizaciones de la economía estocástica en \([0,T]\),
	\(\mathcal{F}\) una \(\sigma\)-álgebra de subconjunto de \(\Omega\),
	\(\{\mathcal{F}_t\}\) una filtración generada por el proceso de precios
	relevante para los activos en la economía.
	
	Según la hipótesis de no-arbitraje, existe una medida martingala
	equivalente \(\mathbb{Q}\) a \(\mathbb{P}\).
	
	Uno de los objetivos es valuar opciones estilo Americano con cash-flows
	aleatorios que pueden ocurrir en \([0,T]\). Se supondrá que los payoffs
	de dichas opciones está en el espacio de variables aleatorias con
	varianza finita.
	
	Se puede demostrar que el valor de una opción Americana se puede
	representar como una cobertura de Snell; el valor de una opción
	Americana es igual al valor máximo del valor presente de los cash-flows
	de dicha opción, donde el máximo se toma sobre todos los tiempos de paro
	con respecto a la filtración.
	
	\textbf{Notación:} Se denotará por \(C(\omega, s; t,T)\) la trayectoria
	de cash-flows generados por la opción, condicional a que la opción no se
	ejerce antes (o hasta) del tiempo \(t\).
	
	Uno de los objetivos del algoritmo LSM es obtener una aproximación por
	trayectorias a la regla de paro óptimo que maximiza el valor de la
	opción Americana. Se supondrá que la opción Americana sólo se puede
	ejercer el \(m\) tiempos discretos
	\(0<t_1\leq t_2\leq \ldots \leq t_m=T\) y se considerará la política de
	paro óptimo en cada fecha de ejercicio. Por supuesto las opciones
	Americana pueden ejercerse en cualquier punto en el tiempo, el algortimo
	LSM se puede aproximar el valor de dichas opciones con \(m\)
	suficientemente grande.
	
	En la fecha de vencimineto el poseedor de la opción decide ejercerla
	sólo si ésta tiene un payoff positivo. Sin embargo, en la fecha de
	ejercicio \(t_j\) previa a la fecha de vencimiento, el poseedor de la
	opción debe decidir si ejercer la opción o continuar con ella y
	replantearse la disyuntiva en la ``siguiente'' fecha de ejercicio.
	
	El valor de la opción se maximiza por trayectorias si el inversionista
	ejerce cuando el valor de ejercicio inmediato es mayor ó igual que el
	valor de continuación.
	
	Al tiempo \(t_j\), el cash-flow de ejercer inmediatamente se conoce (es
	simplemente el ``payoff'' en ese momento). Sin embargo, el cash-flow de
	continuación no se conoce al tiempo \(t_j\). Además, la teoría de
	valuación de no-arbitraje considera al valor de continuación como la
	esperanza del valor presente esperado de los cash-flows restantes
	\(C(\omega,s;t:j,T)\), donde el valor esperado es con respecto a la
	medida riesgo neutro \(\mathcal{Q}\) Es decir, el valor de continuación
	al tiempo \(t_j\), \(V_c(\omega,t_j)\), se puede escribir como
	
	\[V_c(\omega,t_j) = \mathbb{E}_{\mathbb{Q}}\left[\sum_{i=j+1}^m \exp\left\{-\int_{t_j}^{t_i}r(\omega,s)ds\right\}C(\omega,t_i;t_j,T)|\mathcal{F}_{t_j}\right],\]
	
	donde \(r(\omega,t)\) es la tasa libre de riesgo (posiblemente
	estocástica) y la esperanza que se considera es condicional a la
	información \(\mathcal{F}_{t_j}\) al tiempo \(t_j\). Con esta
	representación, el problema de ejercicio óptimo se reduce a comparar el
	ejercicio inmediato con esta esperanza condicional; y entonces ejercer
	cuando el valor de ejercicio sea positivo y mayor ó igual que la
	esperanza condicional.
	
	\subsubsection{El algoritmo LSM}\label{el-algoritmo-lsm}
	
	La metodología LSM utiliza mínimos cuadrados para aproximar la función
	esperanza condicional en \(t_m, t_{m-1},\ldots,t_1\). Se trabaja hacia
	atrás ya que las trayectorias de los cash-flows \(C(\omega,s:t,T)\)
	generadas por la opción se definen de manera recursiva. Puede ocurrir
	que \(C(\omega,s;t_j,T)\) sea diferente de \(C(\omega,s;t_{j+1},T)\) ya
	que puede ser óptimo parar al tiempo \(j_{j+1}\) y por ende cambiar
	todos los cash-flows subsecuentes en toda la trayectoria observada
	\(\omega\). En particular, la tiempo \(t_{m-1}\) se supone que la forma
	funcional desconocida de \(V_c(\omega;t_{m-1})\) se puede representar
	como una combinación lineal de un conjunto numerable de funciones base
	\(\mathcal{F}_{t_{m-1}}\)-medibles, i.e.~variables aleatorias con
	respecto a la \(\sigma\)-álgebra \(\mathcal{F}_{t_{m-1}}\)
	
	Esta suposición se puede justificar de manera formal, con alguna
	hipótesis. Por ejemplo, si la esperanza condicional es una variable
	aleatoria con varianza finita (con respecto a alguna medida), i.e.~está
	en \(L^2\). como \(L^2\) es un espacio de Hilbert, entonces tiene una
	base ortonormal numerable y dicha esperanza condicional se puede
	representar como una función lineal de los elementos de dicha base. Una
	posible elección de las funciones base es el conjunto de polinomios de
	Laguerre ponderados: \(L_0(x) = \exp\{-x/2\}\),
	\(L_1(x)=\exp\{-x/2\}(1-x)\),
	\(L_2(x)=\exp\{-x/2\}(1-2x+\frac{1}{2}x^2)\),\ldots{},
	\(L_n(x)=\exp\{-x/2\}\frac{e^x}{n!}\frac{\partial^2}{\partial x^n}(x^ne^{-x})\).
	
	Con esta especificación \(V_c(\omega,t_{m-1})\) se puede expresar como
	
	\[V_c(\omega,t_{m-1})=\sum_{i=0}^{\infty}a_iL_i(X),\]
	
	donde \(X\) es el valor del activo subyacente de la opción y los
	coeficientes \(a_i\)'s son constantes. Otros tipos de funciones base son
	los polinomios de Hermite, Legendre, Chebyshev, Gegenbauer y Jacobi.
	Algunas pruebas numéricas han probado que series de Fourier y potencias
	simples también dan resultados adecuados.
	
	Para implementar la metodología LSM se aproxima \(V_0(\omega,t_{m-1})\)
	utilizando las primeras \(M\) funciones base. Se denotará por
	\(V_{c,M}(\omega;t_{m-1})\) a esta aproximación.
	\(V_{c,M}(\omega;t_{m-1})\) se estima a partir de hacer regresión de
	\(C(\omega,s;t_{m-1},T)\) con las funciones base como regresores en
	aquellas trayectorias en las que la opción tiene un payoff positivo al
	tiempo \(t_{m-1}\). Sólo se consideran las trayectorias en las que el
	payoff es positivo .
	
	Al sólo considerar las trayectorias con payoff positivo, se restringe la
	región en la que se debe estimar la esperanza condicional y se necesitan
	menor regresores para obtener una aproximación adecuada de la función
	esperanza condicional.
	
	Como los valores de las funciones base son independientes e
	idénticamente distribuidas a lo largo de las trayectorias, no se
	requieren suposiciones mayores para la existencia de los momomentos y
	pot lo tanto, el valor ajustado de esta regresión,
	\(\widehat{V}_{c,M}(\omega,t_{m-1})\) converje en media cuadrática y en
	probabilidad a \(V_{c,M}(\omega,t_{m-1})\) conforme el número de
	trayectorias con payoff positivo en la simulación tiende a infinito.
	Además, \(\widehat{V}_{c,M}(\omega,t_{m-1})\) es el mejor estimador
	lineal insesgado de \(V_{c,M}(\omega,t_{m-1})\) en términos de métrica
	de media cuadrática.
	
	\subsection{Resultados de
		convergencia}\label{resultados-de-convergencia}
	
	Mediante el algoritmo LSM, se tiene un mecanismo sencillo de aproximar
	la estrategia de ejercicio óptimo para una opción estilo Americano (y
	Bermuda también).
	
	Hay algunos resultados de convergencia que garantizas que dicha
	aproximación efectivamente es posible
	
	\textbf{Proposición:} Para cualesquiera \(m,M\in \mathbb{N}_+\) y
	\(\beta\in\mathbb{R}^{m\times(m-1)}\) que representa el vector de
	coeficientes para las \(M\) funciones base en cada uno de las \(m-1\)
	fechas de ejercicio (anticipado), sea \(LSM(\omega;M,m)\) el valor
	presente que se obtiene de la regla LSM en caso de que el valor de
	ejercicio sea positivo y mayor ó igual que
	\(\widehat{V}_{c,M}(\omega_i,t_j)\) (que se define a partir de
	\(\beta\)). Entonces, se satisface
	
	\[V(X)\geq \frac{1}{n}\sum_{i=1}^n LSM(\omega_i;M,m),\ \mbox{casi seguramente},\]
	
	donde \(V(X)\) es el verdadero valor de la opción Americana.
	
	Esta proposición establece que el algoritmo LSM imploca una regla de
	paro para opciones Americanas. Sin embargo el valor de dicha opción
	depende de la regla de paro que maximiza su valor. Otras reglas de paro,
	incluyendo la que se obtiene con el algortimo LSM, llevan a valores
	menores o iguales que la que induce la regla de paro óptimo.
	
	Este resultafo es muy útil pues mediante éste se determina un criterio
	objetivo para establecer algún tipo de convergencia. Con este criterio
	se puede obtener algún parámetro o métrica que permita determinar el
	número de funciones base que se necesitan para obtener una aproximación
	más exacta: incrementar el valor de \(m\) hasta que el precio que se
	obtiene con el algoritmo LSM no presente cambios significativos. Este es
	una mejora algorítmica pues otros algoritmos que simplemente llevan a
	valor presente el valor de continuación a posteriori no permiten este
	monitoreo de la convergencia.
	
	Aunque se quisiera algún resultado de convergencia más fuerte para el
	algorimo LSM, desde su construcción esto es difícil pues se necesita
	considerar límites con respecto al número de discretizaciones \(m\), el
	número de funciones base \(M\) (los regresores) y el número de
	trayectorias simuladas (\(n\rightarrow \infty\)). Además, se debe
	considerar los efectos de la estimación en las reglas de paro desde el
	tiempo \(t_{m-1}\) a \(t_1\).
	
	\textbf{Proposición:} Supóngase que el precio de una opción Americana
	sólo depende de la variable de precio del activo subyacente \(X\) (que
	puede tomar valores en \((0,\infty)\)) cuya dinámica estocástica sigue
	un proceso de Markov. Además, supóngase que sólo se puede ejercer la
	opción en los tiempos \(t_1\) y \(t_2\) y que la función esperanza
	condicional \(V_c(\omega;t_1)\) es absolutamente continua y que las dos
	siguientes integrales existen en \(\mathbb{R}\)
	\[\int_0^{\infty}e^{-x}V^2_c(\omega;t_1)dx, \int_0^{\infty}e^{-x}V^2_X(\omega;t_1)dx\]
	Entonces, para cualquier \(\epsilon>0\), existe \(M\in\mathbb{R}\) tal
	que
	
	\[\lim_{n\rightarrow \infty} \mathbb{P}\left(\left|V(X)-\frac{1}{n}\sum_{i=1}^n LSM(\omega_i;M,m)\right|>\epsilon\right)=0\]
	
	Este resultado garantiza que si se escoge una valor de \(M\)
	suficientemente grande y se hace \(n\rightarrow \infty\), con el
	algoritmo LSM se obtiene un valor para la opción Americana cercano en
	menos de una distancia de \(\epsilon\) del verdadero valor.
	
	\subsection{Implementación de Paro}
	
	\textit{\href{https://www.dropbox.com/s/pfarddbjlv2tc5z/implementacion_paro.pdf?dl=0}{Ver cuaderno de R}}
	
	\subsection{Estimación de Parámetros}
	
	\textit{\href{https://www.dropbox.com/s/njszo4jprkzxhz6/estimacion_parametros_edp.pdf?dl=0}{Ver cuaderno de R}}
	
	\section{Conclusiones}\label{conclusiones}
	

	\begin{thebibliography}{9}
			\bibitem{longstaff} Longstaff, F. \& Schwartz, E. (2001). Valuing American Options by Simulation: A Simple Least-Squares Approach. The Review of Financial Studies Spring 2001 Vol. IS. No. I, pp. 113-147 The Society for Financial Studies
	\end{thebibliography}
	
	%%\end{multicols}
	
	
\end{document}